% CVPR 2024 Paper Template; see https://github.com/cvpr-org/author-kit

\documentclass[10pt,twocolumn,letterpaper]{article}

%%%%%%%%% PAPER TYPE  - PLEASE UPDATE FOR FINAL VERSION
% \usepackage{cvpr}              % To produce the CAMERA-READY version
\usepackage[final]{cvpr}      % To produce the REVIEW version
% \usepackage[pagenumbers]{cvpr} % To force page numbers, e.g. for an arXiv version

% Import additional packages in the preamble file, before hyperref
%
% --- inline annotations
%
\usepackage[dvipsnames]{xcolor}
\newcommand{\red}[1]{{\color{red}#1}}
\newcommand{\todo}[1]{{\color{red}#1}}
\newcommand{\TODO}[1]{\textbf{\color{red}[TODO: #1]}}
% --- disable by uncommenting  
% \renewcommand{\TODO}[1]{}
% \renewcommand{\todo}[1]{#1}



% It is strongly recommended to use hyperref, especially for the review version.
% hyperref with option pagebackref eases the reviewers' job.
% Please disable hyperref *only* if you encounter grave issues, 
% e.g. with the file validation for the camera-ready version.
%
% If you comment hyperref and then uncomment it, you should delete *.aux before re-running LaTeX.
% (Or just hit 'q' on the first LaTeX run, let it finish, and you should be clear).
\definecolor{cvprblue}{rgb}{0.21,0.49,0.74}
\usepackage[pagebackref,breaklinks,colorlinks,citecolor=cvprblue]{hyperref}

%%%%%%%%% PAPER ID  - PLEASE UPDATE
\def\paperID{*****} % *** Enter the Paper ID here
\def\confName{CVPR}
\def\confYear{2024}

%%%%%%%%% TITLE - PLEASE UPDATE
\title{Autoregressive Generation of Neural Field Weights}

%%%%%%%%% AUTHORS - PLEASE UPDATE
\author{Luca Fanselau\\
TUM Munich\\
{\tt\small go68vog@mytum.de}
% For a paper whose authors are all at the same institution,
% omit the following lines up until the closing ``}''.
% Additional authors and addresses can be added with ``\and'',
% just like the second author.
% To save space, use either the email address or home page, not both
\and
Luis Muschal\\
TUM Munich\\
{\tt\small go98zoh@mytum.de}
}

\begin{document}
\maketitle
\begin{abstract}

  % Discussion of problem definition:
  The absence of a clear structure of implicit neural fields (NeF) makes it difficult to
  apply generative modeling directly to synthesize new data.
  % Proposed solution:
  To this end, we propose a novel approach for generating Multi Layer Perceptron (MLP)-weights of neural
  fields in an autoregressive transformer-based fashion.
  % Expected outcome:
  Our approach is expected to unconditionally and autoregressively generate the MLP-weights of
  novel neural fields.




\end{abstract}
\section{Introduction}
\label{sec:intro}

% eleborate on motivation
In recent years we have seen an astonishing success in neural field models, for compressed scene representation, as well as in autoregressive
transformer-based generative models (eg. Large Language Models).
Additionally, first approaches towards generating novel neural fields using diffusion models \cite{erkoç2023hyperdiffusion} and unconditional triangle-mesh generation using transformers \cite{siddiqui2023meshgpt} have been proposed.
% and problem statement
However, the absence of a clear structure of implicit neural fields (NeF) makes it difficult to apply to an autoregressive process, which assumes sequence data. Another Problem is the continuous nature of MLP weights, while transformers typically use a small to medium sized vocabulary.
Therefore, the main problem lies in finding a representation of the MLP weights that can be used to train sequence-to-sequence architectures.
% contribution bullet points

\noindent \textbf{Our contributions are:}
\begin{itemize}
    \item Combine the idea of autoregressive transformer-based generation with neural fields
    \item Quantitatively compare different embedding strategies
\end{itemize}

% Literature Review
\section{Related Works}
\label{sec:literature}

\subsection*{implicit neural fields and Diffusion Models}
Recent advancements have demonstrated the effectiveness of implicit neural fields in representing high-fidelity 3D geometries and radiance fields. For instance, DeepSDF \cite{park2019deepsdflearningcontinuoussigned} encodes the shapes of objects as signed distance functions using a multi-layer neural network, and NeRF uses MLPs to encode radiance fields for photorealistic rendering from novel views \cite{mildenhall2020nerfrepresentingscenesneural}. Additionally, methods like Fourier features and periodic activation functions have been proposed to improve the representation of complex signals by addressing the bias towards learning low-frequency details in standard MLP \cite{tancik2020fourierfeaturesletnetworks, sitzmann2020implicitneuralrepresentationsperiodic}. Our methods aims to generate novel implicit neural fields that represent the implicit signal of both image data as well as the surface of 3D structures.


\subsection*{Transformer-Based 3D Structure Generation}
Transformer architectures have shown promise in generating 3D structures. MeshGPT \cite{siddiqui2023meshgpt}, for example, uses a decoder-only transformer to autoregressively generate triangle meshes, representing them as sequences of geometric embeddings. This approach has demonstrated improvements in mesh generation quality, emphasizing the capability of transformers to handle complex geometric data efficiently. Adapting this technique, our pipeline uses a GPT like transformer but changes the domain from triangle meshes to MLP-weights.


\subsection*{Diffusion Models in Generative Modeling}
Diffusion models have been emerging in the recent past and shown promising results for novel generation tasks. Specifically, HyperDiffusion \cite{erkoç2023hyperdiffusion} operates directly on the MLP weights of neural fields, enabling high-fidelity synthesis of 3D and 4D shapes. This method leverages a transformer-based architecture to model the diffusion process, achieving state-of-the-art performance in generating compact and coherent implicit neural fields. Influenced by this work we show the possibility of generating novel implicit neural field weights using an autoregressive process.

% Literature Review
\section{Proposed Methods}
\label{sec:method}
% explanation of your planned methods
We want to train a transformer-based architecture to generate the MLP-weights of novel
NeF in an autoregressive, unconditioned fashion. This is first done using images as a
proof of concept and then extended to 3D structures.\\

\noindent We propose the following approaches:
\subsection*{Naive Approach}
Directly perform a regression task on the MLP weights of the NeF and use a continuous loss function.
Additionally, we also want to investigate the positional encoding to inform the model about the structure within the MLP.
\subsection*{Learned Embedding}
Since the MLP is a fully connected graph, we want to propose an encoder-decoder architecture that captures the underlying structure in the latent space, by for example using Graph-CNNs. This representation is then used to train the transformer on a regression task.
\subsection*{Learned Vocabulary}
Since transformers usually excel with predicting tokenized sequences we also want to investigate methods to quantize the input, either directly from the weights or the latent-space representation. Instead of a regression task the transformer would predict the probability distribution of the most likely next token.

% Literature Review
\section{Experiments}
\label{sec:exper}
% Datasets
\textbf{Data: }
The experiments leverage the dataset introduced by \cite{papa2023train}, which contains neural radiance fields overfitted on the SIREN architecture using the MNIST, CIFAR10, MicroImageNet and ShapeNet datasets. This allows for evaluation on both 2D image data as well as 3D shape representations. \\
% Metrics and Baselines
\textbf{Metrics and Baselines: }
Evaluating the quality of synthesized neural fields poses challenges due to the lack of ground truth data. We adopt the metrics proposed by \cite{erkoç2023hyperdiffusion}, specifically the Minimum Matching Distance (MMD), Coverage (COV), and 1-Nearest-Neighbor Accuracy (1-NNA), to facilitate comparisons with the seminal works that inspired this research and serve as performance baselines. \\


{
    \small
    \bibliographystyle{ieeenat_fullname}
    \bibliography{../bibliography.bib}
}

% WARNING: do not forget to delete the supplementary pages from your submission 
% \clearpage
\setcounter{page}{1}
\maketitlesupplementary


\section{Rationale}
\label{sec:rationale}
% 
Having the supplementary compiled together with the main paper means that:
% 
\begin{itemize}
\item The supplementary can back-reference sections of the main paper, for example, we can refer to \cref{sec:intro};
\item The main paper can forward reference sub-sections within the supplementary explicitly (e.g. referring to a particular experiment); 
\item When submitted to arXiv, the supplementary will already included at the end of the paper.
\end{itemize}
% 
To split the supplementary pages from the main paper, you can use \href{https://support.apple.com/en-ca/guide/preview/prvw11793/mac#:~:text=Delete%20a%20page%20from%20a,or%20choose%20Edit%20%3E%20Delete).}{Preview (on macOS)}, \href{https://www.adobe.com/acrobat/how-to/delete-pages-from-pdf.html#:~:text=Choose%20%E2%80%9CTools%E2%80%9D%20%3E%20%E2%80%9COrganize,or%20pages%20from%20the%20file.}{Adobe Acrobat} (on all OSs), as well as \href{https://superuser.com/questions/517986/is-it-possible-to-delete-some-pages-of-a-pdf-document}{command line tools}.

\end{document}
