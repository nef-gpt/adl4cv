


\section{Results}

\subsection*{Metrics: }

The ShapeNet medium model was evaluated using the Minimum Matching Distance(MMD), Coverage (COV), and 1-Nearest-Neighbor Accuracy (1-NNA) metrics, and the Frechet PointNet++ Distance as proposed by HyperDiffusion to enable qualitative comparison \cite{erkoç2023hyperdiffusion}.
While for COV higher is better, indicating a full coverage of the reference space, for 1-NNA 50\% is the best possible score.

\begin{gather*}
  \mathrm{MMD}(S_g,S_r) =\frac{1}{|S_{r}|}\sum_{Y\in S_{r}}\min_{X\in S_{g}}D(X,Y)\\
  \text{COV}(S_g,S_r) =\frac{|\{\arg\min_{Y\in S_r}D(X,Y)|X\in S_g\}|}{|S_r|}\\
  \text{1-NNA}(S_g,S_r) =\\
  \frac{\sum_{X\in S_g}\mathbbm{1}[N_X\in S_g]+\sum_{Y\in S_r}\mathbbm{1}[N_Y\in S_r]}{|S_g|+|S_r|}\\
  N_X = \underset{Y\in S_r \cup S_g}{\text{argmin}} \, D(X,Y) \\
\end{gather*}

Similar to the reference work, the Chamfer Distance (CD) for 3D point clouds was used as a distance measure $D(X, Y)$ and multiplied by a constant $10^2$ to enable a comparison with the metrics in the HyperDiffusion paper.

Additionally, a perceptual metric was also adapted, the Frechet PointNet++ Distance (FPD) \cite{qi2017pointnetdeephierarchicalfeature}, which is an extension to the domain of Point Clouds to the Frechet Inception Distance (FID) metric \cite{NIPS2017_8a1d6947}. For the FPD lower values are better.

\subsection*{Model Evaluation}
In Figure \ref{fig:generation} novel generated airplanes are shown. For this the SOS-token is fed into the transformer and the tokens are generated in an autoregressive fashion. The tokens are then back transformed to MLP-weights using the learned codebooks, the reconstructed NeF is used to calculate a point cloud containing SDF-values and the marching cube algorithm is deployed to transform the point cloud to a valid mesh.
A comparison regarding the performance of the model against different diffusion models was conducted as there is no direct comparison model available for other autoregressive techniques in this context. The results are shown in Table \ref{tab:shapenet}.

\begin{tabular}{l c c c}
  \toprule
  Method                   & COV \(\%\) $\uparrow$ & 1-NNA \(\%\) $\downarrow$ & FPD $\downarrow$ \\
  \midrule
  Diffusion Voxel Baseline & 28                    & 94.1                      & 38.9             \\
  PVD                      & 39                    & 76.3                      & 5.8              \\
  DPC                      & 46                    & 74.7                      & 18.7             \\
  HyperDiffusion           & 49                    & 69.3                      & 3.5              \\

  \hline

  Ours [3, 0.8]            & \textbf{38.77}        & 90.77                     & \textbf{31.34}   \\
  Ours [3, 1]              & 34.57                 & 91.31                     & 36.78            \\

  Ours [5, 0.8]            & 37.50                 & \textbf{90.43}            & 35.21            \\
  Ours [5, 1]              & 35.64                 & 91.36                     & 31.87            \\
  \bottomrule
\end{tabular}


The results show that in principle the generation of neural field weights using this novel approach is feasible, as we outperform the diffusion based voxel baseline in Coverage, 1-NNA and output fidelity. We suspect that the performance of the method could be further improved by increasing the model size and training time, as shown in Figure \ref{fig:loss} the training loss is still decreasing after the full training, but this investigation was not feasible due to time and compute limitations.


